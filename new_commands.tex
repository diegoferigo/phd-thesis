\newcommand{\ie}{i.e.\@\xspace}
\newcommand{\eg}{e.g.\@\xspace}
\newcommand{\wrt}{w.r.t.\@\xspace}
\newcommand{\etc}{etc.\@\xspace}
\newcommand{\wrtl}{with respect to\@\xspace}

% prescript
\usepackage{mathtools}

% https://tex.stackexchange.com/questions/4302/prettiest-way-to-typeset-c-cplusplus
\newcommand{\cpp}{C\texttt{++}\xspace}

% https://tex.stackexchange.com/a/56769
\DeclareMathOperator*{\E}{\mathbb{E}}

% https://jblevins.org/log/latex-tips
\DeclareMathOperator{\given}{\mid}
\DeclareMathOperator{\giventall}{\,\middle\vert\,}

\newcommand{\pos}[1][]{\prescript{#1}{}{\mathbf{p}}}
\newcommand{\posd}[1][]{\prescript{#1}{}{\dot{\mathbf{p}}}}
\newcommand{\ori}[1][]{\prescript{#1}{}{\mathbf{o}}}
\newcommand{\orid}[1][]{\prescript{#1}{}{\dot{\mathbf{o}}}}
\newcommand{\rot}[1][]{\prescript{#1}{}{\mathbf{R}}}
\newcommand{\rotd}[1][]{\prescript{#1}{}{\dot{\mathbf{R}}}}
\newcommand{\homo}[1][]{\prescript{#1}{}{\mathbf{H}}}
\newcommand{\homod}[1][]{\prescript{#1}{}{\dot{\mathbf{H}}}}
\newcommand{\veladj}[1][]{\prescript{#1}{}{\mathbf{X}}}
\newcommand{\quat}[1][]{\prescript{#1}{}{\mathtt{Q}}}
\newcommand{\trans}[1][]{\prescript{#1}{}{\mathbf{H}}}
\newcommand{\transvel}[1][]{\prescript{#1}{}{\mathbf{X}}}
\newcommand{\transfor}[1][]{\prescript{}{#1}{\mathbf{X}}}
\newcommand{\velsix}[1][]{\prescript{#1}{}{\mathbf{v}}}
\newcommand{\velsixd}[1][]{\prescript{#1}{}{\dot{\mathbf{v}}}}
\newcommand{\vellin}[1][]{\prescript{#1}{}{\boldsymbol{v}}}
\newcommand{\velang}[1][]{\prescript{#1}{}{\boldsymbol{\omega}}}
\newcommand{\accsix}[1][]{\prescript{#1}{}{\mathbf{a}}}
\newcommand{\acclin}[1][]{\prescript{#1}{}{\boldsymbol{a}}}
\newcommand{\accang}[1][]{\prescript{#1}{}{\boldsymbol{\alpha}}}
\newcommand{\accsixproper}[1][]{\prescript{#1}{}{\mathbf{\bar{a}}}}
\newcommand{\forcesix}[1][]{\prescript{}{#1}{\mathbf{f}}}
\newcommand{\forcelin}[1][]{\prescript{}{#1}{\boldsymbol{f}}}
\newcommand{\forceang}[1][]{\prescript{}{#1}{\boldsymbol{m}}}

\newcommand{\crossvelsix}[1][]{#1\!{\times}}
\newcommand{\crossforsix}[1][]{#1\!\bar{\times}^{*}}

\newcommand{\transrepr}[1][]{\prescript{#1}{}{T}}

\newcommand{\masssix}[1][]{\prescript{}{#1}{\mathbb{M}}}
\newcommand{\gravity}[1][]{\prescript{#1}{}{\mathbf{g}}}

\newcommand{\realn}{\mathbb{R}}
\newcommand{\eye}{\mathbf{I}}
\newcommand{\ones}{\boldsymbol{1}}
\newcommand{\zeros}{\boldsymbol{0}}

\newcommand{\Qu}{\mathbf{q}}
\newcommand{\Qud}{\dot{\mathbf{q}}}
\newcommand{\Shape}{\mathbf{s}}
\newcommand{\shape}{s}
\newcommand{\Shaped}{\dot{\mathbf{s}}}
\newcommand{\Shapedd}{\ddot{\mathbf{s}}}
\newcommand{\Nu}[1][]{\prescript{#1}{}{\boldsymbol{\nu}}}
\newcommand{\Nud}[1][]{\prescript{#1}{}{\dot{\boldsymbol{\nu}}}}
\newcommand{\Torques}{\boldsymbol{\tau}}

\newcommand{\jac}[1][]{\prescript{#1}{}{J}}

\newcommand{\subspace}{\mathbf{S}}
\newcommand{\subspacee}[1][]{\prescript{#1}{}{\subspace}}

\newcommand{\jax}{\spacedlowsmallcaps{JAX}\xspace}
\newcommand{\jaxsim}{\spacedlowsmallcaps{JAX}sim\xspace}
\newcommand{\scenario}{Scenar\spacedlowsmallcaps{IO}\xspace}

\newcommand{\SO}{\text{SO}}
\newcommand{\so}{\mathfrak{so}}
\newcommand{\SE}{\text{SE}}
\newcommand{\se}{\mathfrak{se}}

\newcommand{\pid}{\spacedlowsmallcaps{PID}\xspace}

% https://tex.stackexchange.com/questions/5223/command-for-argmin-or-argmax
% \DeclareMathOperator*{\argmin}{arg\,min} % thin space, limits underneath in displays
\DeclareMathOperator*{\argmin}{argmin} % no space, limits underneath in displays
% \DeclareMathOperator{\argmin}{arg\,min} % thin space, limits on side in displays
% \DeclareMathOperator{\argmin}{argmin} % no space, limits on side in displays
% \DeclareMathOperator*{\argmax}{arg\,max} % thin space, limits underneath in displays
\DeclareMathOperator*{\argmax}{argmax} % no space, limits underneath in displays
% \DeclareMathOperator{\argmax}{arg\,max} % thin space, limits on side in displays
% \DeclareMathOperator{\argmax}{argmax} % no space, limits on side in displays

% https://www.overleaf.com/learn/latex/Theorems_and_proofs
% https://tex.stackexchange.com/a/327085
\usepackage{amsthm}

\theoremstyle{definition}
\newtheorem{definition}{Definition}[section]
\newenvironment{definition*}
  {\pushQED{\qed}\definition}
  {\popQED\enddefinition}

\newtheorem{example}{Example}[section]
\newenvironment{example*}
  {\pushQED{\qed}\example}
  {\popQED\endexample}

\newtheorem{remark}{Remark}[section]
\newenvironment{remark*}
  {\pushQED{\qed}\remark}
  {\popQED\endremark}

\newtheorem{theorem}{Theorem}[section]
\newenvironment{theorem*}
  {\pushQED{\qed}\theorem}
  {\popQED\endtheorem}

\newtheorem{assumption}{Assumption}[section]
\newenvironment{assumption*}
  {\pushQED{\qed}\assumption}
  {\popQED\endassumption}
