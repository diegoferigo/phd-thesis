%*******************************************************
% Table of Contents
%*******************************************************

\phantomsection
\pdfbookmark[1]{\contentsname}{tableofcontents}

\setcounter{tocdepth}{1} % <-- 2 includes up to subsections in the ToC
\setcounter{secnumdepth}{2} % <-- 3 numbers up to subsubsections
\manualmark
\markboth{\spacedlowsmallcaps{\contentsname}}{\spacedlowsmallcaps{\contentsname}}

\tableofcontents

\vfill
\begin{center}
    \textbf{Word Count: 57000}
\end{center}

\automark[section]{chapter}
\renewcommand{\chaptermark}[1]{\markboth{\spacedlowsmallcaps{#1}}{\spacedlowsmallcaps{#1}}}
\renewcommand{\sectionmark}[1]{\markright{\textsc{\thesection}\enspace\spacedlowsmallcaps{#1}}}

%*******************************************************
% List of Figures and of the Tables
%*******************************************************
\clearpage
% \pagestyle{empty} % Uncomment this line if your lists should not have any headlines with section name and page number
\begingroup
    \let\clearpage\relax
    \let\cleardoublepage\relax

    %*******************************************************
    % List of Figures
    %*******************************************************

    \phantomsection
    %\addcontentsline{toc}{chapter}{\listfigurename}
    \pdfbookmark[1]{\listfigurename}{lof}
    \listoffigures

    \vspace{8ex}

    %*******************************************************
    % List of Tables
    %*******************************************************

    \newpage
    \phantomsection
    %\addcontentsline{toc}{chapter}{\listtablename}
    \pdfbookmark[1]{\listtablename}{lot}
    \listoftables

    \vspace{8ex}
    % \newpage

    %*******************************************************
    % List of Listings
    %*******************************************************

    % \phantomsection
    % %\addcontentsline{toc}{chapter}{\lstlistlistingname}
    % \pdfbookmark[1]{\lstlistlistingname}{lol}
    % \lstlistoflistings

    % \vspace{8ex}

    %*******************************************************
    % Acronyms
    %*******************************************************

    \phantomsection
    \pdfbookmark[1]{Acronyms}{acronyms}
    %\markboth{\spacedlowsmallcaps{Acronyms}}{\spacedlowsmallcaps{Acronyms}}
    \chapter*{Acronyms}
    \begin{acronym}[UMLX]
        \acro{ABA}{Articulated Body Algorithm}
        \acro{AD}{Automatic Differentiation}
        \acro{AI}{Artificial Intelligence}
        \acro{API}{Application Programming Interface}
        \acro{CH}{Convex Hull}
        \acro{CoM}{Center of Mass}
        \acro{CoP}{Center of Pressure}
        \acro{CDF}{Cumulative Density Function}
        \acro{CP}{Capture Point}
        \acro{CPU}{Central Processing Unit}
        \acro{CRBA}{Composite Rigid Body Algorithm}
        \acro{DCM}{Divergent Component of Motion}
        \acro{DL}{Deep Learning}
        \acro{DoF}{Degree of Freedom}
        \acrodefplural{DoF}[DoFs]{Degrees of Freedom}
        \acro{DRL}{Deep Reinforcement Learning}
        \acro{DS}{Double Support}
        \acro{EoM}{Equation of Motion}
        \acrodefplural{EoM}[EoMs]{Equations of Motion}
        \acro{F/T}{Force/Torque}
        \acro{GAE}{Generalized Advantage Estimator}
        \acro{GPU}{Graphics Processing Unit}
        \acro{GUI}{Graphical User Interface}
        \acro{HAL}{Hardware Abstraction Layer}
        \acro{IMU}{Inertial Measurement Unit}
        \acro{JIT}{Just-in-time}
        \acro{KL}{Kullback–Leibler}
        \acro{LIP}{Linear Inverted Pendulum}
        \acro{MDP}{Markov Decision Process}
        \acrodefplural{MDP}[MDPs]{Markov Decision Processes}
        \acro{MJCF}{Mujoco XML Format}
        \acro{ML}{Machine Learning}
        \acro{NN}{Neural Network}
        \acro{ODE}{Ordinary Differential Equation}
        \acro{PDF}{Probability Density Function}
        \acro{PG}{Policy Gradient}
        \acro{PPO}{Proximal Policy Optimization}
        \acro{RK4}{Runge-Kutta 4}
        \acro{RNG}{Random Number Generator}
        \acro{RBDA}{Rigid Body Dynamics Algorithm}
        \acro{RBF}{Radial Basis Function}
        \acro{RTF}{Real-Time Factor}
        \acro{RL}{Reinforcement Learning}
        \acro{RNEA}{Recursive Newton-Euler Algorithm}
        \acro{SAC}{Soft Actor-Critic}
        \acro{SDF}{Simulation Description Format}
        \acro{SP}{Support Polygon}
        \acro{TPU}{Tensor Processing Unit}
        \acro{TRPO}{Trust Region Policy Optimization}
        \acro{URDF}{Unified Robot Description Format}
        \acro{USD}{Universal Scene Description}
        \acro{XLA}{Accelerated Linear Algebra}
        \acro{ZMP}{Zero Moment Point}
    \end{acronym}

    %*******************************************************
    % List of Symbols
    %*******************************************************

    \newpage
    \phantomsection
    \pdfbookmark[1]{List of Symbols}{los}
    
    \nomenclature[A, 01]{$\pos$}{Point in space}
    \nomenclature[A, 01]{$A, B, C, \dots$}{Frame names}
    \nomenclature[A, 01]{$W$}{World (inertial) frame}
    \nomenclature[A, 02]{$[A]$}{Orientation frame of frame A}
    \nomenclature[A, 02]{$\ori_A \in \realn^3$}{Origin of frame A}
    \nomenclature[A, 02]{$A = (\ori_A, [A])$}{Definition of frame A}
    \nomenclature[A, 03]{$\pos[A] \in \realn^3$}{Coordinate vector of point $\pos$ expressed in frame $A$}
    \nomenclature[A, 03]{$\rot[A]_B \in \SO(3)$}{Rotation matrix from orientation frame $[B]$ to $[A]$}
    \nomenclature[A, 04]{$\homo[A]_B \in \SE(3)$}{Homogeneous transformation from frame $B$ to $A$}
    \nomenclature[A, 04]{${}^{A}\tilde{\mathbf{p}} \in \realn^4$}{Homogeneous representation of coordinate vector $\pos[A]$}
    \nomenclature[A, 05]{$\vellin[C]_{A,B} \in \realn^3$}{Linear velocity of frame $B$ relative to frame $A$, expressed in $C$}
    \nomenclature[A, 06]{$\velang[C]_{A,B} \in \realn^3$}{Angular velocity of frame $B$ relative to frame $A$, expressed in $C$}
    \nomenclature[A, 06]{$\velsix[C]_{A,B} \in \realn^6$}{6D velocity of frame $B$ relative to frame $A$, expressed in $C$}
    \nomenclature[A, 06]{$\velsix[C]_{A,B}^\wedge \in \se(3)$}{Matrix representation of 6D velocity $\velsix[C]_{A,B}$}
    \nomenclature[A, 07]{$\transvel[A]_B \in \realn^{6\times 6}$}{Velocity transformation from frame $B$ to frame $A$}
    \nomenclature[A, 08]{$\velsix[C]_{A,B} \crossvelsix \in \realn^{6\times6}$}{Cross product operator on $\realn^6$ for 6D velocities}
    \nomenclature[A, 09]{$\velsixd[C]_{A,B} \in \realn^3$}{Apparent acceleration of frame $B$ relative to $A$, expressed in C}
    \nomenclature[A, 10]{$\accsix[C]_{A,B} \in \realn^3$}{Intrinsic acceleration of frame $B$ relative to $A$, expressed in C}
    \nomenclature[A, 10]{$\accsixproper[C]_{A,B} \in \realn^3$}{Proper acceleration of frame $B$ relative to $A$, expressed in C}
    \nomenclature[A, 10]{$\forcelin[A] \in \realn^3$}{Linear force expressed in frame $A$}
    \nomenclature[A, 11]{$\forceang[A] \in \realn^3$}{Angular force (torque) expressed in frame $A$}
    \nomenclature[A, 11]{$\forcesix[A] \in \realn^6$}{6D force expressed in frame $A$}
    \nomenclature[A, 11]{$\transfor[A]^B \in \realn^{6\times 6}$}{Force transformation from frame $B$ to frame $A$}
    \nomenclature[A, 11]{$\velsix[C]_{A,B} \crossforsix \in \realn^{6\times6}$}{Cross product operator on $\realn^6$ for 6D forces}
    \nomenclature[A, 11]{$I \in \realn^{3\times3}$}{Inertia tensor}
    \nomenclature[A, 12]{$\masssix[B] \in \realn^{6\times6}$}{6D inertia matrix, expressed in frame $B$}
    \nomenclature[A, 12]{$g \in \realn^+$}{Standard gravity}
    \nomenclature[A, 13]{$\gravity[W] \in \realn^3$}{Gravitational acceleration vector}
    \nomenclature[A, 13]{$\subspacee[X]_{P,C} \in \realn^6$}{Joint motion subspace between frame $P$ and $C$, expressed in $X$}
    \nomenclature[A, 13]{$n \in \mathbb{N}$}{Number of degrees of freedom of a multibody system}
    \nomenclature[A, 14]{$\Qu \in \SE(3) \times \realn^n$}{Floating-base position}
    \nomenclature[A, 15]{$\Nu[X] \in \realn^{6+n}$}{Floating-base velocity having base velocity $\velsix[X]_{W,B}$}
    \nomenclature[A, 16]{$\Shape \in \realn^n$}{Joint positions}
    \nomenclature[A, 16]{$\Shaped \in \realn^n$}{Joint velocities}
    \nomenclature[A, 16]{$\Shapedd \in \realn^n$}{Joint accelerations}
    \nomenclature[A, 16]{$\jac[Y]_{B,E} \in \realn^{6\times n}$}{Relative Jacobian of frame $E$ \wrt $B$, expressed in $Y$}
    \nomenclature[A, 17]{$\jac[Y]_{W,E} \in \realn^{6\times (6+n)}$}{Free-floating Jacobian of frame $E$, expressed in $Y$}
    \nomenclature[A, 18]{$\jac[Y]_{W,E/X} \in \realn^{6\times (6+n)}$}{Free-floating Jacobian of frame $E$, expressed in $Y$, for base velocity expressed in $X$}
    \nomenclature[A, 19]{$M(\Qu) \in \realn^{(6+n)\times (6+n)}$}{Mass matrix}
    \nomenclature[A, 20]{$C(\Qu, \Nu) \in \realn^{(6+n)\times(6+n)}$}{Coriolis matrix}
    \nomenclature[A, 20]{$g(\Qu) \in \realn^{6+n}$}{Potential force vector (or gravity vector)}
    \nomenclature[A, 20]{$h(\Qu, \Nu) \in \realn^{6+n}$}{Bias forces vector}
    \nomenclature[A, 21]{$\Torques \in \realn^n$}{Joint generalized forces}
    \nomenclature[A, 22]{$\mathcal{L}$}{Set of link indices}
    \nomenclature[A, 23]{$n_L$}{Number of links}
    \nomenclature[A, 24]{$\forcesix_\mathcal{L}^\text{ext} \in \realn^{6 \times n_L}$}{Vector stacking external forces applied to all links}
    \nomenclature[A, 24]{$\jac_\mathcal{L} \in \realn^{6 n_L \times (6+n)}$}{Matrix stacking floating-base Jacobians of all links}
    \nomenclature[A, 25]{$q \in \mathbb{H}$}{A quaternion}
    \nomenclature[A, 26]{$\bar{q} \in \operatorname{Spin}(3)$}{A unit quaternion}
    \nomenclature[A, 27]{$\quat = (w, \mathbf{r}) \in \realn^4$}{Quaternion coefficients}
    \nomenclature[A, 28]{$\mathcal{H}$}{Function providing terrain height}
    \nomenclature[A, 29]{$\mathcal{S}$}{Function providing terrain normal}
    \nomenclature[A, 30]{$\mathbf{m}$}{3D tangential deformation of the terrain's material}
    \nomenclature[A, 31]{$(\cdot)^\parallel, \, (\cdot)_\parallel$}{Component parallel to the terrain}
    \nomenclature[A, 31]{$(\cdot)^\perp, \, (\cdot)_\perp$}{Component normal to the terrain}
        
    \nomenclature[L, 01]{$\mathcal{S}$}{The state space}
    \nomenclature[L, 02]{$\mathcal{A}$}{The action space}
    \nomenclature[L, 03]{$s_t \in \mathcal{S}$}{State of the environment at time $t$}
    \nomenclature[L, 04]{$a_t \in \mathcal{A}$}{Action applied to the environment at time $t$}
    \nomenclature[L, 05]{$\mathcal{P} \,:\, \mathcal{S} \times \mathcal{A} \to \operatorname{Pr}[\mathcal{S}]$}{State-transition probability density function}
    \nomenclature[L, 06]{$\mathcal{R} \,:\, \mathcal{S} \times \mathcal{A} \times \mathcal{S} \to \realn$}{Reward function}
    \nomenclature[L, 07]{$r_t \in \realn$}{Immediate reward at time $t$}
    \nomenclature[L, 08]{$a = \mu(s)$}{Action taken from deterministic policy in state $s$}
    \nomenclature[L, 09]{$a \sim \pi(\cdot \given s)$}{Action sampled from stochastic policy in state $s$}
    \nomenclature[L, 10]{$\pi_{\boldsymbol{\theta}}$}{Stochastic policy parameterized with $\boldsymbol{\theta}$}
    \nomenclature[L, 11]{$\tau = (s_0, a_0, s_1, a_1, \dots, s_T)$}{Trajectory of states and actions}
    \nomenclature[L, 12]{$s_0 \sim \rho_0(\cdot)$}{Sampling state from initial state distribution}
    \nomenclature[L, 13]{$R_t$}{Return at time $t$}
    \nomenclature[L, 13]{$\hat{R}_t$}{Reward-to-go at time $t$}
    \nomenclature[L, 14]{$R(\tau)$}{Discounted return of trajectory $\tau$}
    \nomenclature[L, 15]{$J(\pi)$}{Performance function of stochastic policy $\pi$}
    \nomenclature[L, 16]{$\pi^*$}{Optimal policy}
    \nomenclature[L, 17]{$\langle \mathcal{S}, \mathcal{A}, \mathcal{R}, \mathcal{P}, \mathcal{\rho}_0 \rangle$}{Tuple defining a Markov Decision Process}
    \nomenclature[L, 18]{$V^\pi(s)$}{State-value function for policy $\pi$ at state $s$}
    \nomenclature[L, 18]{$Q^\pi(s, a)$}{Action-value function for policy $\pi$ at state-action pair $(s, a)$}
    \nomenclature[L, 19]{$A^\pi(s, a)$}{Advantage function for policy $\pi$ at state-action pair $(s, a)$}
    \nomenclature[L, 20]{$\mathbb{E}[\cdot]$}{Expected value of a random variable}
    \nomenclature[L, 21]{$\hat{\mathbb{E}}[\cdot]$}{Empirical average estimating the expected value of a random variable from samples}

    \renewcommand{\nomname}{List of Symbols}
    \printnomenclature

\endgroup
