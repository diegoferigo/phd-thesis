\chapter{Center of Pressure}
\label{appendix:center_of_pressure}

\begin{figure}
    \centering
    \resizebox{\textwidth}{!}{
    \includegraphics{images/appendix/center_of_pressure.tikz}
    }
    \caption{Illustration of the setting in which the \ac{CoP} is calculated.}
    \label{fig:center_of_pressure}
\end{figure}

In this section, we provide the intuition and equations for computing the \ac{CoP} of a rigid body.
A practical example is the computation of the \ac{CoP} of a robot foot modelled as a body having the shape of a box, as illustrated in Figure~\ref{fig:center_of_pressure}.
Let us assume the following properties:
%
\begin{itemize}
    \item We assume a flat terrain with a normal $\hat{n} = (0, 0, 1) \in \realn^3$. Therefore, the normal remains constant over the entire surface supporting the body.
    \item We consider the existence of $k$ external 6D forces $\forcesix_1,\, \forcesix_2,\, \dots, \, \forcesix_k \in \realn^6$ applied to different points of the body.
    \item We introduce an unknown frame $C = (\pos[W]_{CoP}, [W])$ corresponding to the \ac{CoP} with $\pos[W]_{CoP}$ belonging to the supporting surface. The flat terrain assumption implies that the $z$ axis of the $C$ frame is normal to the terrain.
\end{itemize}
%
\begin{definition*}[Center of Pressure]
    The \ac{CoP} is the point $\pos[W]_{CoP} \in \realn^3$ belonging to the rigid body's supporting surface to which a pure linear force $\forcelin[W]_{CoP} \in \realn^3$ can be applied to produce the equivalent effects along the normal direction of all the external 6D forces.
\end{definition*}
%
As first step, we need to combine all the external forces, computing a single \emph{equivalent} 6D force $\forcesix[W]_{eq} \in \realn^6$.
If the external forces are expressed in $W$, the equivalent force is just the sum of external forces $\forcesix[W]_{eq} = \sum_{i} \forcesix[W]_i$.
Instead, if they are are expressed in a local frame $F_i = (\pos[W]_i, [W])$, assuming the knowledge of the application point, we can use the transform introduced in Equation~\eqref{eq:sixd_force_transformation}:
%
\begin{equation*}
    \forcesix[W]_{eq} =
    \begin{bmatrix}
        \forcelin[W]_{eq} \\ \forceang[W]_{eq}
    \end{bmatrix}
    =s
    \sum_{i} \transfor[W]^{F_i} \forcesix[{F_i}]_i
    .
\end{equation*}
%
We can project the equivalent 6D force into the normal and tangential components with respect ground:
%
\begin{equation*}
    \forcesix[W]_{eq}
    =
    \begin{bmatrix}
        \forcelin[W]_{eq}^\perp + \forcelin[W]_{eq}^\parallel \\
        \forceang[W]_{eq}
    \end{bmatrix}
    =
    \begin{bmatrix}
        \forcelin[W]_{eq}^\perp \\
        \forceang[W]_{eq}
    \end{bmatrix}
    +
    \begin{bmatrix}
        \forcelin[W]_{eq}^\parallel \\
        \zeros_3
    \end{bmatrix}
    =
    \forcesix[W]_{eq}^\perp + \forcesix[W]_{eq}^\parallel
    ,
\end{equation*}
%
where we grouped the equivalent angular component $\forceang[W]_{eq}$ with the perpendicular term.
The linear component of the equivalent perpendicular force can be computed as:
%
\begin{equation*}
    \forcelin[W]_{eq}^\perp = (\forcelin[W]_{eq} \cdot \hat{n}) \, \hat{n} =
    \begin{bmatrix}
        0 \\ 0 \\ f_{eq}^{\perp z}
    \end{bmatrix}
    .
\end{equation*}
%
The point associated to the \ac{CoP} can now be obtained by solving for $\pos[W]_{CoP}$ the following equation that expresses the equivalent 6D force into the \ac{CoP} and enforces it to be a pure linear force by setting the resulting angular component to zero:
%
\begin{align*}
    \forcesix[C]_{CoP} &= \begin{bmatrix}
        \forcelin[C]_{CoP} \\ \zeros_3
    \end{bmatrix} =
    \transfor[C]^W \forcesix[W]_{eq}^\perp =
    \begin{pmatrix}
        \eye_3 & \zeros_3 \\
        -\pos[W]_{CoP}^\wedge & \eye_3
    \end{pmatrix}
    \begin{bmatrix}
        \forcesix[W]_{eq}^\perp \\ \forceang[W]_{eq}
    \end{bmatrix} \\
    &=
    \begin{bmatrix}
        \forcesix[W]_{eq}^\perp \\
        \forcesix[W]_{eq}^\perp \times \pos[W]_{CoP} + \forceang[W]_{eq}
    \end{bmatrix}
    .
\end{align*}
%
where we used the relation $\ori[C]_W = -\pos[W]_{CoP}$.
We can rearrange the second equation as follows:
%
\begin{equation*}
    \forceang[W]_{eq} =
    \begin{bmatrix}
        m_{eq}^{x} \\
        m_{eq}^{y} \\
        m_{eq}^{z}
    \end{bmatrix} =
    \forcelin[W]_{eq}^\perp \times \pos[W]_{CoP} =
    \begin{pmatrix}
        0 & - f_{eq}^{\perp z} & 0 \\
        f_{eq}^{\perp z} & 0 & 0 \\
        0 & 0 & 0
    \end{pmatrix}
    \begin{bmatrix}
        p_{CoP}^{x} \\
        p_{CoP}^{y} \\
        p_{CoP}^{z}
    \end{bmatrix}
\end{equation*}
%
and finally resolve for $\pos[W]_{CoP}$:
%
\begin{equation*}
    \pos[W]_{CoP} =
    \begin{bmatrix}
        m_{eq}^{y} / f_{eq}^{\perp z} \\
        -m_{eq}^{x} / f_{eq}^{\perp z} \\
        0
    \end{bmatrix}
    .
\end{equation*}
%
This result can be extended to different terrain normals through geometrical transformations.
The most impacting difference would regard the frame corresponding to the \ac{CoP}, defined as $C = (\pos[W]_{CoM}, [W])$.
If the terrain has a different normal, the orientation of the frame should still be considered a known value but, this time, obtained by the properties of the surface.
