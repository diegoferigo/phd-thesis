\newpage
\section*{Covid-19 Impact Statement}

In late December 2019, mysterious pneumonia characterised by fever, dry cough, and fatigue started spreading with remarkable speed in the city of Wuhan, China\footnote{Wu, Yi-Chia et al., "The outbreak of COVID-19: An overview", 2020.}.
%
Aided by the globalisation that characterises our modern society, the disease causing such symptoms first travelled rapidly to nearby countries and, in just a few weeks, world-wide to the most densely populated urban centers.
%
On 13 March 2020, the World Health Organization declared the novel disease, caused by a new coronavirus, a pandemic.
Humankind was pushed into a global crisis, with unique characteristics in recent history, at least considering its spatial extent, rapid onset, and complexity of consequences\footnote{Cheval, S. et al., "Observed and Potential Impacts of the COVID-19 Pandemic on the Environment", 2020}.
%
The first outbreak of COVID-19 in Italy occurred during the second half of February 2020 in some areas in the North of the country.
Due to the high contagiousness of the infection, further spread by asymptomatic people, Italy has become in a few weeks the country with the largest number of infected people in the world\footnote{Megna, R., "First month of the epidemic caused by COVID-19 in Italy: current status and real-time outbreak development forecast", 2020.}.

The pandemics affected, without any exception, the life of each and every human being.
In attempt to contain the spread of the disease, which became overwhelming for most national health systems,  countries started enforcing lock-downs of varying stringency.
The host institution of this research project, the Italian Institute of Technology located in Genova, Italy, was closed from March to June 2020.
In the following period, the research institute enforced a fixed quota of persons having access to the laboratories and equipment.

Originally, the research project was structured in two stages: the first stage mainly focused on simulated results, and the second on practical activities with real robots, addressing sim-to-real applications.
Before the beginning of the pandemic, the first activities of the research project were directed towards the development of a software infrastructure, presented in Chapter~\ref{ch:rl_env_for_robotics}, to enable running experiments both on simulated and real robots.
The most acute phase of the pandemic occurred in the beginning of the second year of the research project, when we were achieving with the developed architecture the first results in simulation that are presented in Chapter~\ref{ch:learning_from_scratch}.
In the same year, I was also personally affected by the disease, which forced me to stay home for 20 days.
Due to the limitations in accessing the laboratory, and the uncertainty of how the pandemic would have evolved in conjunction with new corresponding restrictions that could have been enforced nationwide, we decided to refocus the research project to rely only on activities that could have been performed and finalised without requiring physical access to the facilities.
In particular, we updated the remaining objectives of the research project to study methodologies to mitigate the sampling bottlenecks of \acl{RL} architectures applied to robot learning applications.
In this setting, the training methods typically gather synthetic data from rigid body simulators, which is later fed into an optimisation problem to synthesise the control action.
Despite the need to train multiple \aclp{NN}, and the computational cost of the optimisation algorithms, the sampling process often represents the major bottleneck of the complete learning pipeline.
For this reason, we decided to study in greater detail how synthetic data could be efficiently sampled from rigid body simulations, maintaining the focus on robot locomotion.
Chapter~\ref{ch:contact_aware_dynamics} and Chapter~\ref{ch:scaling_rigid_body_simulations} will present the outputs of these final objectives.
